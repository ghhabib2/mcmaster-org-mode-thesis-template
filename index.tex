% Created 2021-03-06 Sat 01:52
% Intended LaTeX compiler: pdflatex
\documentclass[12pt]{report}

\usepackage{gscale_thesis_doublespace}
\usepackage{fancyheadings}
\usepackage{natbib}
\usepackage{setspace}
\title{My Thesis}
\halftitle{Ph.D. Thesis title} % 60 Characters Max. Including Spaces

\author{Author's Name}
\shortauthor{A.I} % Author's initials going to be  used for page header

\dept{Your department name} % The department you are part of; Must be all lower case
\field{Domain/Sub domain of the thesis} % What field your thesis is in

\prevdegreeone{Ph.D. (Domain \& Sub domain),\\ McMaster University, Hamilton, Canada}
\prevdegreetwo{Ph.D.} % Just your degree's field
% Define the kind of Thesis. 
\degree{Ph.D. of Applied Science }
\submitdate{May 2021}
\copyrightyear{2021}

\principaladviser{Prof. First Supervisor, Prof. Second Supervisor} % Your Supervisor/Supervisors

\setcounter{tocdepth}{1}
\usepackage{graphicx}
\usepackage{subcaption}
\usepackage[justification=centering]{caption}
\usepackage[hidelinks]{hyperref}
\usepackage{array}
\newcolumntype{P}[1]{>{\raggedright\let\newline\\\arraybackslash\hspace{0pt}}m{#1}}
\usepackage{booktabs}
\usepackage{longtable}
\usepackage{float}
\usepackage{enumerate}
\usepackage[shortlabels]{enumitem}
\usepackage[shortcuts]{extdash}
\usepackage{amsmath}
\usepackage{amsfonts}
\usepackage{xcolor}
\numberwithin{equation}{section}
\author{Habib Ghaffari Hadigheh}
\date{\today}
\title{My Ph.D. Thesis Title}
\hypersetup{
 pdfauthor={Habib Ghaffari Hadigheh},
 pdftitle={My Ph.D. Thesis Title},
 pdfkeywords={},
 pdfsubject={Documentation of my Ph.D. Thesis},
 pdfcreator={Emacs 27.1 (Org mode 9.4.4)}, 
 pdflang={English}}
\begin{document}

\maketitle
\tableofcontents

\#\#+OPTIONS:  tex:t toc:nil title:nil

\beforepreface                                         % Half title page, title page, declaration page   
  \include{layabstr}                                  % Lay Abstract
  \include{abstr}                                      % Abstract
  \include{dedic}                                      % Dedication
  \include{acknowledgements}                 % Acknowledgements
  \referencepageswithnotations{notation} % Table of Contents, List of Figures, List of Tables, Notations
  %\referencepages                                 % No notations version (choose one)
\afterpreface
\chapter{Introduction}
\label{sec:orge371da8}
Every thesis needs an introductory chapter

\setcounter{figure}{0}
\setcounter{equation}{0}x3
\setcounter{table}{0}i
\chapter{Your Chapter Title}
\label{cp1}
This is a sample chapter. For the sake of labelling, I assume this is the first
chapter of your thesis. As you can see I define the chapter using one \texttt{*}. After
writing the chapter name you need to define the chapter properties. Here where
we define the chapter label as \texttt{:CUSTOM\_ID:}. The \texttt{cp\#} is abbreviation for
\textbf{Chapter}. In order to cross references the chapter \ref{cp1} using link definition
in org mode. In order to reference the chapter you should use \texttt{\#} follow by the
label name inside link definition as it can be seen above.

\section{Referencing}
\label{cp1:s1}
Here we defined a new section for the Chapter \ref{cp1}. As it can be seen, we use
the abbreviation \texttt{s} for labelling the section follow by section number. For the
cross-referencing of sections, follow the same instruction for
cross-referencing of  chapters.

These are some sample references to GAMYGDALA \citep{popescu2014gamygdala} from
the \texttt{references.bib} file and state effects of cognition
\citep{hudlicka2002time} from the same file. In the latex template, these
references are in separate \texttt{.bib} files, while here we merge two files
together for convenience.

\section{Figures}
\label{cp1:s2}
This is a single image figure (Figure \ref{cp1:s2:fig1}). You can label the figure
using \texttt{\#+name}. The latex related attributes for the figure should written using
\texttt{+\#attr\_latex}. Finally caption for the figure should be written using
\texttt{\#+caption}. Figure \ref{cp1:s2:fig1} depicts an example figure and its org-mode
definition.

\begin{figure}[!ht]
\centering
\includegraphics[width=0.6\textwidth]{./figures/Sample/tumblr_static_eaceks0rfxsss8o4swscw40wo.jpg}
\caption{\label{cp1:s2:fig1}This is a single figure environment}
\end{figure}


I haven't find any solution to display to images side by side using org-mode.
But fortunately it is still possible to solve this problem using the pure \texttt{latex}
coding embedded to an \texttt{org} file. Here is the example

You can also define multi-image figure using latex sub-figure definition.
Figure\ref{cp1:s2:fig2} depicts an example of multi-image figure.

\begin{figure}[ht]
	\centering
	\begin{subfigure}[t]{\textwidth}
		\centering
		\includegraphics[width=0.7\textwidth]{figures/Sample/tumblr_static_eaceks0rfxsss8o4swscw40wo.jpg}
		\caption{Figure 1}
		\label{fig_multienv_1}
	\end{subfigure}
	~
	\begin{subfigure}[t]{\textwidth}
		\centering
		\includegraphics[width=0.7\textwidth]{figures/Sample/tumblr_static_eaceks0rfxsss8o4swscw40wo.jpg}
		\caption{Figure 2}
		\label{fig_multienv_2}
	\end{subfigure}
	
	\caption{A Multi-Figure Environment}
	\label{cp1:s2:fig2}
\end{figure}

\begin{verbatim}
\begin{figure}[ht]
  \centering
  \begin{subfigure}[t]{\textwidth}
    \centering
    \includegraphics[width=0.7\textwidth]{Image URL}
    \caption{Figure 1}
    \label{fig_multienv_1}
  \end{subfigure}
  ~
  \begin{subfigure}[t]{\textwidth}
    \centering
    \includegraphics[width=0.7\textwidth]{Image URL}
    \caption{Figure 2}
    \label{fig_multienv_2}
  \end{subfigure}

  \caption{A Multi-Figure Environment}
  \label{cp1:s2:fig2}
\end{figure}
\end{verbatim}

For the side by side figure in \texttt{HTML} you can use the code bellow

\section{Tables}
\label{cp1:s3}
Here is a sample table coded using table builder of the \texttt{org-mode}. You can use
\texttt{ATTR\_LATEX} to set the different attributes of the table (Table \ref{cp1:s2:tbl1}):

\begin{table}[!ht]
\caption[Sample Table]{\label{cp1:s2:tbl1}This is table's long caption A table sample}
\centering
\begin{tabular}{m{0.2\textwidth}  m {0.1\textwidth} m{0.15\textwidth}}
\toprule
A & \(\longleftrightarrow\) & B\\
C & \(\longleftrightarrow\) & D\\
\bottomrule
\end{tabular}
\end{table}

You can also directly embed the \texttt{latex} code inside the \texttt{org} file. Here is the
example of typing latex table inside the the org file. The code bellow will
generate the table depicts above.

\begin{verbatim}
#+name: cp1:s2:tbl1
#+attr_latex: :width \textwidth :placement [!ht]
#+caption: A table sample
\begin{table}
  \centering
  \begin{tabular}{ m{0.2\textwidth} m {0.1\textwidth} m{0.15\textwidth} }
    \toprule
    A & $\longleftrightarrow$ & B \\
    C & $\longleftrightarrow$ & D \\
    \bottomrule	
  \end{tabular}	
\end{table}
\end{verbatim}

You can use \texttt{latex} base table definition for this purpose as it depicted above.

\subsection{Long Tables}
\label{cp1:s3:ss1}
A sample long table is shown in \href{appendixB.org}{Appendix B}  where we described the application of
long table.

\section{Equations}
\label{cp1:s4}
Here is a sample equation (Equation \ref{cp1:s4:eq1}):

\begin{equation}
\label{cp1:s4:eq1}
	y = mx + b
\end{equation}
\setcounter{figure}{0}
\setcounter{equation}{0}
\setcounter{table}{0}
\chapter{Conclusion}
\label{conclusion}
Every thesis also needs a concluding chapter
\setcounter{figure}{0}
\setcounter{equation}{0}
\setcounter{table}{0}

\begin{appendix}
\chapter{Your Appendix}
\label{appendix_a}
Your appendix goes here.
\setcounter{figure}{0}
\setcounter{equation}{0}
\setcounter{table}{0}
\chapter{Long Tables}
\label{appendix_b}
This appendix demonstrates the use of a long table that spans multiple pages.

\begin{longtable}{P{3cm}P{3cm}P{2.5cm}P{3.5cm}}
\toprule
\midrule
\textbf{Col A} & \textbf{Col B} & \textbf{Col C} & \textbf{Col D}\\
\midrule
\endfirsthead
\multicolumn{4}{l}{Continued from previous page} \\
\toprule

\textbf{Col A} & \textbf{Col B} & \textbf{Col C} & \textbf{Col D} \\

\midrule
\endhead
\midrule\multicolumn{4}{r}{Continued on next page} \\
\endfoot
\endlastfoot
A & B & C & D\\
\midrule
A & B & C & D\\
\midrule
A & B & C & D\\
\midrule
A & B & C & D\\
\midrule
A & B & C & D\\
\midrule
A & B & C & D\\
\midrule
A & B & C & D\\
\midrule
A & B & C & D\\
\midrule
A & B & C & D\\
\midrule
A & B & C & D\\
\midrule
A & B & C & D\\
\midrule
A & B & C & D\\
\midrule
A & B & C & D\\
\midrule
A & B & C & D\\
\midrule
A & B & C & D\\
\midrule
A & B & C & D\\
\midrule
A & B & C & D\\
\midrule
A & B & C & D\\
\midrule
A & B & C & D\\
\midrule
A & B & C & D\\
\bottomrule
\end{longtable}

As it stated before, you can generate the same long table by embedding the
\texttt{latex} code inside \texttt{org} file. Here is the example of what the final results
will be.

\begin{verbatim}
\begin{center}
\begin{longtable}{P{3cm}P{3cm}P{2.5cm}P{3.5cm}}
\toprule
\hline
\textbf{Col A} & \textbf{Col B} & \textbf{Col C} & \textbf{Col D} \\ \midrule

\endfirsthead
\multicolumn{4}{c}{\textit{Continued from previous page}} \\ \hline
\textbf{Col A} & \textbf{Col B} & \textbf{Col C} & \textbf{Col D} \\ \hline
\endhead
\hline \multicolumn{4}{r}{\textit{Continued on the next page}} \\
\endfoot
\hline
\endlastfoot

A & B & C & D \\ \midrule

A & B & C & D \\ \midrule

A & B & C & D \\ \midrule

A & B & C & D \\ \midrule

A & B & C & D \\ \midrule

A & B & C & D \\ \midrule

A & B & C & D \\ \midrule

A & B & C & D \\ \midrule

A & B & C & D \\ \midrule

A & B & C & D \\ \midrule

A & B & C & D \\ \midrule

A & B & C & D \\ \midrule

A & B & C & D \\ \midrule

A & B & C & D \\ \midrule

A & B & C & D \\ \midrule

A & B & C & D \\ \midrule

A & B & C & D \\ \midrule

A & B & C & D \\ \midrule

A & B & C & D \\ \midrule

A & B & C & D \\ \midrule

\hline
\end{longtable}
\end{center} 
\end{verbatim}
\setcounter{figure}{0}
\setcounter{equation}{0}
\setcounter{table}{0}
\end{appendix}

\bibliographystyle{natbib}
\bibliography{references} 

\label{NumDocumentPages}
\end{document}