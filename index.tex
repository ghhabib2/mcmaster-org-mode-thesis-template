% Created 2022-03-29 Tue 01:45
% Intended LaTeX compiler: pdflatex
\documentclass[12pt]{report}
\usepackage[mathletters]{ucs}
\usepackage{gscale_thesis_doublespace}
\usepackage{fancyheadings}
\usepackage[square,numbers]{natbib}
\usepackage[english]{babel}
\usepackage{setspace}
\title{My Thesis}
\halftitle{Ph.D. Thesis title} % 60 Characters Max. Including Spaces

\author{Author's Name}
\shortauthor{A.I} % Author's initials going to be  used for page header

\dept{Your department name} % The department you are part of; Must be all lower case
\field{Domain/Sub domain of the thesis} % What field your thesis is in

\prevdegreeone{Ph.D. (Domain \& Sub domain),\\ McMaster University, Hamilton, Canada}
\prevdegreetwo{Ph.D.} % Just your degree's field
% Define the kind of Thesis. 
\degree{Ph.D. of Applied Science }
\submitdate{May 2021}
\copyrightyear{2021}

\principaladviser{Prof. First Supervisor, Prof. Second Supervisor} % Your Supervisor/Supervisors

\setcounter{tocdepth}{1}
\usepackage{lscape}
\usepackage{graphicx}
\usepackage{subcaption}
\usepackage[justification=centering]{caption}
\usepackage[hidelinks]{hyperref}
\usepackage{array}
\newcolumntype{P}[1]{>{\raggedright\let\newline\\\arraybackslash\hspace{0pt}}m{#1}}
\usepackage{booktabs}
\usepackage{longtable}
\usepackage{float}
\usepackage{enumerate}
\usepackage[shortlabels]{enumitem}
\usepackage[shortcuts]{extdash}
\usepackage{amsmath}
\usepackage{amsfonts}
\usepackage{amssymb}
\usepackage{makeidx}
\makeindex
\usepackage[xindy]{glossaries}
\makeglossaries
\usepackage{xcolor}
\numberwithin{equation}{section}
\usepackage[mathletters]{ucs}
\usepackage[utf8x]{inputenc}
\usepackage{amsthm}
\newacronym{cpu}{CPU}{Central Processing Unit}
\newglossaryentry{dsl}{name=DSL,description={{A domain-specific language (DSL) is a computer language specialized to a particular application domain}}}
\theoremstyle{definition}
\newtheorem{definition}{Definition}[chapter]
\theoremstyle{remark}
\newtheorem*{remark}{Remark}
\newtheorem{theorem}{Theorem}[chapter]
\newtheorem{corollary}{Corollary}[theorem]
\newtheorem{lemma}[theorem]{Lemma}
\theoremstyle{example}
\newtheorem{example}{Example}
\theoremstyle{axiom}
\newtheorem{axiom}{Axiom}[chapter]
\DeclareMathOperator*{\argmax}{arg\,max}
\DeclareMathOperator*{\argmin}{arg\,min}
\usepackage{euscript}
\newcommand*\euler[1]{\ensuremath{\EuScript{#1}}}
\def \M{\euler M}
\def \L{\euler L}
\def \N{\euler N}
\newcommand{\norm}[1]{\left\Vert#1\right\Vert}
\newcommand{\abs}[1]{\left\vert#1\right\vert}
\newcommand{\set}[1]{\left\{#1\right\}}
\newcommand{\Real}{\mathbb R}
\newcommand{\eps}{\varepsilon}
\newcommand{\To}{\longrightarrow}
\newcommand{\BX}{\mathbf{B}(X)}
\newcommand{\A}{\mathcal{A}}
\author{Frist Name and Last Name}
\date{\today}
\title{My Ph.D. Thesis Title}
\hypersetup{
 pdfauthor={Frist Name and Last Name},
 pdftitle={My Ph.D. Thesis Title},
 pdfkeywords={},
 pdfsubject={Documentation of my Ph.D. Thesis},
 pdfcreator={Emacs 27.2 (Org mode 9.5.1)}, 
 pdflang={English}}
\begin{document}





\beforepreface                                         % Half title page, title page, declaration page   
  \include{layabstr}                                  % Lay Abstract
  \include{abstr}                                      % Abstract
  \include{dedic}                                      % Dedication
  \include{acknowledgements}                 % Acknowledgements
  \referencepages                                 % Table of Contents, List of Figures, List of Tables,
\afterpreface


\chapter{Introduction}
\label{intro}
Every thesis needs an introductory chapter

\setcounter{figure}{0}
\setcounter{equation}{0}
\setcounter{table}{0}

\chapter{Your Chapter Title}
\label{cp1}
This is a sample chapter. For the sake of labelling, I assume this is the first
chapter of your thesis. As you can see I define the chapter using one \texttt{*}. After
writing the chapter name you need to define the chapter properties. Here where
we define the chapter label as \texttt{:CUSTOM\_ID:}. The \texttt{ref:cp\#} is abbreviation for
\textbf{Chapter}. In order to cross references the chapter \ref{cp1} using link definition
in org mode.

\section{Referencing}
\label{cp1:s1}
Here we defined a new section for the Chapter \ref{cp1}. As it can be seen, we use
the abbreviation \texttt{s} for labelling the section follow by section number. For the
cross-referencing of sections, follow the same instruction for
cross-referencing of  chapters.

These are some sample references to GAMYGDALA \cite{popescu2014gamygdala} from
the \texttt{index.bib} file and state effects of cognition
\cite{hudlicka2002time} from the same file. In the latex template, these
references are in separate \texttt{.bib} files, while here we merge two files
together for convenience. However, references should be defined at the end of
this document using the instruction.

\section{Figures}
\label{cp1:s2}
This is a single image figure (Figure \ref{cp1:s2:fig1}). You can label the figure
using \texttt{\#+name}. The latex related attributes for the figure should written using
\texttt{+\#attr\_latex}. Finally caption for the figure should be written using
\texttt{\#+caption}. Figure \ref{cp1:s2:fig1} depicts an example figure and its org-mode
definition.

\begin{figure}[!ht]
\centering
\includegraphics[width=0.8\textwidth]{./figures/Sample/tumblr_static_eaceks0rfxsss8o4swscw40wo.jpg}
\caption{\label{cp1:s2:fig1}This is a single figure environment}
\end{figure}


I haven't find any solution to display to images side by side using org-mode.
But fortunately it is still possible to solve this problem using the pure \texttt{latex}
coding embedded to an \texttt{org} file. Here is the example

You can also define multi-image figure using latex sub-figure definition.
Figure \ref{cp1:s2:fig2} depicts an example of multi-image figure.

\begin{figure}[ht]
	\centering
	\begin{subfigure}[t]{\textwidth}
		\centering
		\includegraphics[width=0.7\textwidth]{figures/Sample/tumblr_static_eaceks0rfxsss8o4swscw40wo.jpg}
		\caption{Figure 1}
		\label{fig_multienv_1}
	\end{subfigure}
	~
	\begin{subfigure}[t]{\textwidth}
		\centering
		\includegraphics[width=0.7\textwidth]{figures/Sample/tumblr_static_eaceks0rfxsss8o4swscw40wo.jpg}
		\caption{Figure 2}
		\label{fig_multienv_2}
	\end{subfigure}
	
	\caption{A Multi-Figure Environment}
	\label{cp1:s2:fig2}
\end{figure}
\begin{verbatim}
\begin{figure}[ht]
  \centering
  \begin{subfigure}[t]{\textwidth}
    \centering
    \includegraphics[width=0.7\textwidth]{Image URL}
    \caption{Figure 1}
    \label{fig_multienv_1}
  \end{subfigure}
  ~
  \begin{subfigure}[t]{\textwidth}
    \centering
    \includegraphics[width=0.7\textwidth]{Image URL}
    \caption{Figure 2}
    \label{fig_multienv_2}
  \end{subfigure}

  \caption{A Multi-Figure Environment}
  \label{cp1:s2:fig2}
\end{figure}
\end{verbatim}
For the side by side  \texttt{HTML} version you can use the code bellow. It is
worthful to state that, we assumed all of the images are going to be saved in
\texttt{figure} folder. The code generates what can be see above this paragraph.

\begin{verbatim}
<div  class="figure">
<div style="width:50%; float:left">
  <p>
    <img src="./figures/[image_url]" width="80%"  alt="[image_alt text]">
  </p>
  <p><span class="figure-number">(a): </span>caption for figure a</p>
</div>
<div style="width:50%; float:left">
  <p>
    <img src="./figures/[image_url]" width="80%"  alt="[image_alt text]">
  </p>
  <p><span class="figure-number">(b):</span>Caption for figure b</p>
</div>
<div style="width:100%">
<p><span class="figure-number">Figure #</span> Caption for figure</p>
</div>
</div>
\end{verbatim}

\section{Tables}
\label{cp1:s3}
Here is a sample table coded using table builder of the \texttt{org-mode}. You can use
\texttt{ATTR\_LATEX} to set the different attributes of the table (Table \ref{cp1:s2:tbl1}):

\begin{table}[!ht]
\caption[Sample Table]{\label{cp1:s2:tbl1}This is table's long caption A table sample}
\centering
\begin{tabular}{m{0.2\textwidth}  m {0.1\textwidth} m{0.15\textwidth}}
\toprule
A & \(\longleftrightarrow\) & B\\
C & \(\longleftrightarrow\) & D\\
\bottomrule
\end{tabular}
\end{table}

You can also directly embed the \texttt{latex} code inside the \texttt{org} file. Here is the
example of typing latex table inside the the org file. The code bellow will
generate the table depicts above.

\begin{verbatim}
#+name: cp1:s2:tbl1
#+attr_latex: :width \textwidth :placement [!ht]
#+caption: A table sample
\begin{table}
  \centering
  \begin{tabular}{ m{0.2\textwidth} m {0.1\textwidth} m{0.15\textwidth} }
    \toprule
    A & $\longleftrightarrow$ & B \\
    C & $\longleftrightarrow$ & D \\
    \bottomrule	
  \end{tabular}	
\end{table}
\end{verbatim}

You can use \texttt{latex} base table definition for this purpose as it depicted above.


\subsection{Long Tables}
\label{cp1:s3:ss1}
A sample long table is shown in \ref{appendix_b}  where we described the application of
long table.


\section{Equations}
\label{cp1:s4}
Here is a sample equation (Equation \ref{cp1:s4:eq1}):

\begin{equation}
\label{cp1:s4:eq1}
	y = mx + b
\end{equation}


\setcounter{figure}{0}
\setcounter{equation}{0}
\setcounter{table}{0}
\section{Acronyms and Glossaries}
\label{cp1:s5}
An abbreviation is a short form of a word or phrase that is usually made by
deleting certain letters. In the following sentence, everything underlined is an
abbreviation.

Acronyms are usually formed using the first letter (or letters) of each word in
a phrase. he first time you use the term, put the acronym in parentheses after
the full term. Thereafter, you can stick to using the acronym. For example, if
it is the first time you are introducing the \acrshort{cpu}, you need to use it as \Acrfull{cpu}. You can also have glossaries as more descriptive
acronyms. As an example I've added \gls{dsl} as a glossary here which is going to
introduce the Domain Specific Languages.


\chapter{Conclusion}
\label{conclusion}
Every thesis also needs a concluding chapter



\setcounter{figure}{0}
\setcounter{equation}{0}
\setcounter{table}{0}

\begin{appendix}


\chapter{Your Appendix}
\label{appendix_a}
Your appendix goes here.

\setcounter{figure}{0}
\setcounter{equation}{0}
\setcounter{table}{0}

\chapter{Long Tables}
\label{appendix_b}
This appendix demonstrates the use of a long table that spans multiple pages.

\begin{longtable}{P{3cm}P{3cm}P{2.5cm}P{3.5cm}}
\toprule
\midrule
\textbf{Col A} & \textbf{Col B} & \textbf{Col C} & \textbf{Col D}\\
\midrule
\endfirsthead
\multicolumn{4}{l}{Continued from previous page} \\
\toprule

\textbf{Col A} & \textbf{Col B} & \textbf{Col C} & \textbf{Col D} \\

\midrule
\endhead
\midrule\multicolumn{4}{r}{Continued on next page} \\
\endfoot
\endlastfoot
A & B & C & D\\
\midrule
A & B & C & D\\
\midrule
A & B & C & D\\
\midrule
A & B & C & D\\
\midrule
A & B & C & D\\
\midrule
A & B & C & D\\
\midrule
A & B & C & D\\
\midrule
A & B & C & D\\
\midrule
A & B & C & D\\
\midrule
A & B & C & D\\
\midrule
A & B & C & D\\
\midrule
A & B & C & D\\
\midrule
A & B & C & D\\
\midrule
A & B & C & D\\
\midrule
A & B & C & D\\
\midrule
A & B & C & D\\
\midrule
A & B & C & D\\
\midrule
A & B & C & D\\
\midrule
A & B & C & D\\
\midrule
A & B & C & D\\
\bottomrule
\end{longtable}

As it stated before, you can generate the same long table by embedding the
\texttt{latex} code inside \texttt{org} file. Here is the example of what the final results
will be.

\begin{verbatim}
\begin{center}
\begin{longtable}{P{3cm}P{3cm}P{2.5cm}P{3.5cm}}
\toprule
\hline
\textbf{Col A} & \textbf{Col B} & \textbf{Col C} & \textbf{Col D} \\ \midrule

\endfirsthead
\multicolumn{4}{c}{\textit{Continued from previous page}} \\ \hline
\textbf{Col A} & \textbf{Col B} & \textbf{Col C} & \textbf{Col D} \\ \hline
\end head
\hline \multicolumn{4}{r}{\textit{Continued on the next page}} \\
\endfoot
\hline
\endlastfoot

A & B & C & D \\ \midrule

A & B & C & D \\ \midrule

A & B & C & D \\ \midrule

A & B & C & D \\ \midrule

A & B & C & D \\ \midrule

A & B & C & D \\ \midrule

A & B & C & D \\ \midrule

A & B & C & D \\ \midrule

A & B & C & D \\ \midrule

A & B & C & D \\ \midrule

A & B & C & D \\ \midrule

A & B & C & D \\ \midrule

A & B & C & D \\ \midrule

A & B & C & D \\ \midrule

A & B & C & D \\ \midrule

A & B & C & D \\ \midrule

A & B & C & D \\ \midrule

A & B & C & D \\ \midrule

A & B & C & D \\ \midrule

A & B & C & D \\ \midrule

\hline
\end{longtable}
\end{center} 
\end{verbatim}


\setcounter{figure}{0}
\setcounter{equation}{0}
\setcounter{table}{0}
\end{appendix}


\printglossaries

\bibliographystyle{abbrvnat}
\bibliography{index}

\label{NumDocumentPages}
\end{document}